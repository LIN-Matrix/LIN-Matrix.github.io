\documentclass{article}
\usepackage{graphicx}
\usepackage{amsmath}
\usepackage{hyperref}

\title{操作系统宏内核网络管理模块接口的设计与实现}
\author{}
\date{}

\begin{document}

\maketitle

\begin{abstract}
简要介绍研究背景、研究目标、设计思路、实现方案及实验验证结果。
\end{abstract}

\textbf{关键词}: 组件化操作系统;宏内核;ArceOS;starry-next;系统调用;轻量级虚拟化

\section{第1章 引言}

\subsection{1.1 研究背景与意义}

\subsection{1.2 国内外研究现状}

\subsection{1.3 研究目标与内容}

\subsection{1.4 论文结构安排}

\section{第2章 ArceOS操作系统架构分析}

\subsection{2.1 ArceOS设计理念:库操作系统与组件化}

\subsection{2.2 ArceOS总体架构及关键组件}

\subsection{2.3 ArceOS中宏内核相关组件的扩展方式}

\subsection{2.4 与starry-next的接口适配关系}

(主要参考贾越凯师兄的博士论文第4章,但性能优化的部分我目前打算省略)

\section{第3章 starry-next系统架构与组件对接}

\subsection{3.1 starry-next的总体结构分析}

\begin{itemize}
    \item 系统初始化流程
    \item 任务调度与内存管理概述
    \item 文件系统与IO子系统简介
\end{itemize}

\subsection{3.2 与ArceOS的模块对接接口分析}

\begin{itemize}
    \item 系统调用接口层(syscall)
    \item 用户态ELF加载机制
    \item 页表与地址空间适配
\end{itemize}

\subsection{3.3 与本论文实现相关模块的功能分析}

- [根据我开发移植的 axnet/src/smoltcp、axnet/src/lwip的细节]

\section{第4章 支持Linux应用的宏内核组件设计与实现}

\subsection{4.1 设计目标与总体思路}

\begin{itemize}
    \item 面向Linux应用兼容性
    \item 保持组件间解耦
\end{itemize}

\subsection{4.2 系统调用模块设计}

\begin{itemize}
    \item syscall编号与处理流程
    \item 多架构支持
\end{itemize}

\subsection{4.3 关键子系统组件实现}

\begin{itemize}
    \item 文件IO接口封装
    \item 网络接口抽象(api/imp/net.rs)
    \item 信号机制与进程管理支持
\end{itemize}

\subsection{4.4 与现有框架的集成适配策略}

\begin{itemize}
    \item API桥接与抽象层设计
    \item 模块注册与调度关系解析
\end{itemize}

\section{第5章 实现与实验评估}

\subsection{5.1 开发与调试环境搭建}

\begin{itemize}
    \item QEMU虚拟机平台
    \item 构建流程、镜像制作与测试工具链
\end{itemize}

\subsection{5.2 测例设计与功能验证}

\begin{itemize}
    \item 基于Linux用户态应用的功能测试
    \item ELF程序运行与系统调用验证
\end{itemize}

\subsection{5.3 性能测试与结果分析}

\begin{itemize}
    \item 系统启动时间对比
    \item 系统调用延迟分析
    \item 内存开销评估
\end{itemize}

\section{第6章 遇到的问题与解决方案}

\subsection{6.1 接口适配中遇到的兼容性问题}

\subsection{6.2 构建过程中的依赖与模块耦合问题}

\subsection{6.3 功能测试中Bug追踪与修复策略}

\subsection{6.4 代码优化与重构记录}

\section{第7章 总结与展望}

\subsection{7.1 本文工作总结}

\subsection{7.2 存在的不足}

\subsection{7.3 后续研究方向}

\section{参考文献}
包含 ArceOS 相关论文、starry-next源代码说明、组件化操作系统领域研究等。

\section{附录}
源码关键模块、启动日志、测例输出结果截图等。

\end{document}
